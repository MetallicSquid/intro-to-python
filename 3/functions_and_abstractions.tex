\documentclass{article}
\usepackage[utf8]{inputenc}
\usepackage[english]{babel}
\usepackage[cache=false]{minted}

\title{Functions and Abstractions in Python}
\author{Guillaume Macneil}

\begin{document}
\maketitle

\begin{abstract}
This is the second lesson on the 'Introduction to Python' course. This course is loosely based around the 'MIT Introduction to Computer Science - Fall 2016' course. The course is more focused towards the Python programming side of things.
\end{abstract}

\section{Abstraction and Decomposition}

Abstraction is an important idea in the realm of programming. Abstraction (in programming) is the concept of using a block of code without necessarily knowing exactly how it works, only knowing how to use it. \medskip

Decomposition goes hand-in-hand with abstraction as it is the act of breaking up your code into reusable sections called \textit{functions}, which can be used later on in your program. \medskip

Abstraction and decomposition can be used together in order to achieve self-contained, reusable and coherent code that hides the minutia and tedious detail in favour of higher-level \textit{functions} and \textit{classes} (we will discuss classes later). \medskip

\begin{minted}{python}
def is_even(int_input):
    return int_input % 2 == 0

is_even(6)
# Returns 'True'
is_even(3)
# Returns 'False'
\end{minted}

There are already multiple things of note in this example. Firstly, the \textit{'def'} is the precursor to writing a function, it tells the interpreter that you are going to \textit{define} a function. Secondly is the \textit{'is\_even'} part, this is the name of the function, this is what you will \textit{call} when you want to use the function. Thirdly, the \textit{'(int\_input):'} section lists the arguments or parameters that will be used by the function, these are the function's inputs. In this example we have one argument, 'int\_input', all arguments are placed inside the two parentheses and the colon tells the interpreter that the next indented block of code is the content of the function. Finally, the \textit{'return'} defines the output of the function, which in this case is the Boolean output of whether the input is divisible by 2 without any remainder. \medskip

Underneath the function, you can see that it was called twice with the values of 6 and 3. 6 and 3 were the \textit{int\_inputs} and the output was whether they were even or not. \medskip

\begin{minted}{python}
def print_smiles(smile_num):
    for i in range(smile_num):
	print(":)")

print_smiles(14)
# Prints ":)" 14 times vertically
\end{minted}

"Well hang on a second!" you might be thinking, "There's no \textit{return} here, only a \textit{print()}". Well... You're right, both \textit{print()} and \textit{return} can be used, but they do have slight differences.
\begin{center}
\begin{tabular}{l|c|c}
	Output type: & \textbf{return} & \textbf{print} \\
	\hline
	Usage: & inside functions & inside and outside functions \\
	Frequency: & can only be used once & can be used multiple times \\
	Effects: & code afterwards is not exectuted & code afterwards is executed \\
	Returned to: & function caller & the console
\end{tabular}
\end{center}

\begin{minted}{python}
def squared(a):
    return a**2

def is_even(int_input):
    return int_input % 2 == 0

# Provided that the output of one is valid as the input of another,
# functions can be chained:

number = 4
is_even(squared(number))
# Returns True

squared(is_even(4))
# Returns "1" (This is a strange consequence of True also being equal to 1)
\end{minted}

As you can see from the above example, the output of certain functions can also be used as the input of another function (provided that you use \textit{return}). An important thing to remember is that the output of one function \textbf{has to match the desired input type} of the function you want to chain it to. \medskip

\begin{minted}{python}
a = 5
b = 3

def complex_operation():
    return ((a ** b) * (a - b))/a + b

complex_operation()
# Returns "53.0"
\end{minted}

As you can see from the above example, functions can access external variables that are not given as arguments (\textit{a} and \textit{b} in this case) and can also not have any arguments. \textbf{It is important to note that external variables cannot be defined inside functions.} Global variables can be used, but I will not teach them to you and generally you should not use them.

\begin{minted}{python}
def is_even(a):
    return a % 2 == 0

def multiple_of_5_and_even(a):
    if is_even(a) and a % 5 == 0:
	print(f"{a} is both even and a multiple of 5.")
    else:
	print(f"{a} is not even and a multiple of 5.")

multiple_of_5_and_even(10)
# Prints "10 is both even and a multiple of 5."
\end{minted}

This example illustrates that functions can be called inside functions, with \textit{is\_even()} being called inside \textit{multiple\_of\_5\_and\_even()}. Also, functions can even be defined inside other functions, but this is rarely very useful.

\end{document}
